\section{Introduction}\label{introduction}

With the advent of digital musical instruments in the 20th Century there
has been a decoupling between interface and sound generation when
playing. Consequently, aspects of the sensory experience of playing have
been lost, namely the sensation of vibration and mechanical impedance
that a traditional musical instrument provides.

The project look at restoring tactile and haptic response of historical
harpsichords for an exhibition at the musical instrument Museo di San
Colombano\footnote{https://genusbononiae.it/en/san-colombano-tagliavini-collection/}
in Bologna. San Colombano is host to the Tagliavini Collection, a
collection of approximately 70 historical keyboard instruments in
working order \cite{Tagliavini2007} gifted to the museum by\\
from Luigi Ferdinando Tagliavini in 2010
\cite{SanColombano2010,Carlino2010}.

Some items among the Tagliavini collection are no longer in a playing
condition or are too fragile to continue to be available by museum
visitors. One of those fragile instruments is a 1547 Harpsichord in the
italian style by Alessandro Trasuntino
\cite{Wraight2024}\footnote{examples of similar instruments by
Trasuntino available online in the Royal College of Music Collection
(1531: \url{https://museumcollections.rcm.ac.uk/collection/Details/collect/58}) and
the Musée de la musique collection
(1538: \url{https://carmentis.kmkg-mrah.be:443/eMP/eMuseumPlus?service=ExternalInterface&module=collection&objectId=106088&viewType=detailView})
}, which was used as the basis for an interactive exhibition commisioned
by San Colombano.

Project was also designed to allow others to easily recreate and iterate
on the designs.

First a small cross-sectional model was used as a proof of concept. Once
a sensing system was decided upon, a larger 49-key scale model was
commissioned with and then built by luthier Roberto Livi. The models are
designed to provide only a tactile and haptic response for the player. A
sensor system was installed into the model and messages were sent to
computer that triggered sounds from a sample library which the player
listens to through headphones. The paper looks at the design decisions,
processes, and considerations to be made when undertaking a project of
this kind.

Overview of fabrication methods, techniques that can be employed.
Finally a discussion of the limitations and areas in which the core
project can be improved and developed further.

In this context, a `traditional' instrument is one whose interface for
generating sound and the sound generating mechanism are fundamentally
coupled. A piano requires a key press which triggers a hammer, which
strikes a string which vibrates a body and air cavity. This system is
tied together and cannot be separated. A synthesizer, be it a theremin,
a control voltage synthesizer, a MIDI controller, these are all
instruments whose interface are decoupled from their sound generation.
This decoupling presents a problem that then needs to be addressed. How
do we re-introduce this dimension to the experience of playing a musical
instrument? How do we, and is it possible to, parameterise and preserve
this experience? This is central problem that is trying to be solved in
\cite{Nichols2002, Timmermans2020, McAlpine2014, Baldwin2016}.

This is particularly pertinent as it relates to conservation of musical
instruments, as they represent both historic objects, but also tools for
creating music. Historical musical instruments present an interesting
problem for cultural heritage as they are both a physical object but
also a tool for making music. This begs the question, can we use haptics
to preserve experiences of the past? The proposed project intends to
develop further the methodologies implemented in musical haptics
projects \cite{MusicalHaptics2018} such as \cite{Timmermans2020}.

Systems exist already for the conversion of piano keyboard. The Yamaha
disklavier, the Don Buchla designed MOOG Piano Bar, Bosendorfer, PNOscan
by QRS focus on reproduction. These same systems could be applied to a
harpsichord, but the limitation is a single data stream per key.

\subsection{Motivations}\label{motivations}

The project was carried out in collaboratoiun with San Colombano. The
results of the project would be presented as an interactive exhibition
for the public.

\begin{itemize}
\item
  existing sysetms

  \begin{itemize}
  \item
    Yamaha Disklavier used as a basis for exploration into haptics and
    the piano keyabord
    \cite{MusicalHaptics2018_04, MusicalHaptics2018_05,
    MusicalHaptics2018_13}
  \item
    Bosendorfer: LED photo transistor pair \cite{Moog1990} for obtaining
    performance data \cite{MusicalHaptics2018_05}

    \begin{itemize}
    \item
      290SE
    \item
      CEUS
    \end{itemize}
  \item
    Don Buchla / Moog Piano Bar
  \item
    PNOscan MIDI 9 QRS Music as MIDI Piano system \cite{McPherson2013}

    \begin{itemize}
    \item
      PNOmation
    \end{itemize}
  \item
    Piano Disc
  \end{itemize}
\end{itemize}

\section{Related Work}\label{related-work}

Methodologies for increasing visitor engagmemnet or

Museum staff want visitors to be engaged but limits in staffing and
funding limit how vistors can interact with collections
\cite{Templeton2018,
McAlpine2014}. For musical instrument museums there is an additional
difficulty in the balance of keeping historical instruments playable
\cite{McAlpine2014}. The fragility and decay of musical instruments
means that there is an inveitable point where instruments will no longer
be in a playable condition \cite{McAlpine2014, Fritz2017}

McAlpine outlines a similar situation to the Tagliavini collection in
his work with the Benton Fletcher Collection \cite{McAlpine2014}. There
was a stipulation by Benton Fletcher that the instruments remain
available to play when they were gifted to National Trust Fenton House.
A custom MIDI interface arranged in a two manual harpischord style which
triggered samples of the instruments in the collection using an
appropraite recording strategy for each. One problem highlighted in user
tests by McAlpine is that the weighted keys did no provide ``an
experiential sense of interacting with a historical keyboard''
\cite{McAlpine2014}. The MIDI interface created from commercially
available weighted keys. McAlpine posits a haptic keyboard augemented
with ``actuators to provide positionally-sensitive real-time force
feedback at point-of-contact'' in a similar approach as the piano
mechanism by Gillespie \cite{Gillespie1996}.

Previous NIME project ``Tromba Moderna'' \cite{Baldwin2016} looked to
avoid the complex engineering problem by simply recreating the tromba
marina and augmenting it. In the case of the Tromba Modern, a piezo
transducer was connected to a sound synthesis engine connected to a
driver inside the instrument to simulate the vibration that would be
expected of an historical tromba marina.

The original design was to take a similar approach to the project by
Timmermans \cite{Timmermans2020}, which was an extension of the system
by Gillespie \cite{Gillespie1996}. The process laid out in Timmermans
project was to begin with a single key model of the mechanism. We began
with a 3-key Model Harpsichord Mechanism by Graziano Bandini
\textbackslash figure\{\}. Before we continued with augementing a model
with actuators it was decided that it we should first validate whether a
recreation of the harpsichord mechanism would suffice in allowing
visitors to suspend disbelief.

Benefits of putting effort into the maintenance of one interface with a
transferable sensor system

An additonal factor not faced in the Tromba Moderna project is the
problem of scale in managing 98 rather than a single data stream of
data.

Material costs mitigated by creating a scaled version of the sound
board, providing only the length of string necessary under comparable
tension and thus resistance to the jack quill.

Finally, a further extension to the projects above was a commmitement to
open sourcing all aspects of the project, hardware, software, and data.

\section{Design}\label{design}

\subsection{Design Problem}\label{design-problem}

Central philosophy of the Tagliavini Collection is that the instrument
should remain playable to the public. The condition of some instruments
is such that restoration would required a replacement of vast majority
of the instrument, would be too costly, would risk the instrument given
its fragility, or all of those in combination.

This is not an uncommon problem faced by musical instrument museums
worldwide. The central question the project tries to answer is ``how do
we conserve musical instruments in order to continue interacting with
them''

The approach taken was to recreate the interface of these instruments
not for acoustics purposes be purely for the mechanical resoponse. As
exhibition for the general public in a musical instruments there were
aesthetic and functional constraints placed on the design of the
interface.

The design criteria was to create a keyboard interface that provides a
comparable physical response to a harpsichord.

\subsubsection{Design Constraints}\label{design-constraints}

\begin{itemize}
\item
  no electronics should be visible.
\item
  it should be robust and reliable, easily maintained by museum staff
  without complex technical processes
\item
  it should not augment or fix the mechanics, but present them with
  their original limitation
\end{itemize}

Another constraint was that the design of the sensor system and the
model keyboard would have to carried out at a distance in separate
workshops. Time constraints meant the keyboard was being fabricated for
a sensor that did not yet exist.

Accomodations were made in the design to allow for as much flexibility
as possible. This also meant that designs for the eletronics would have
to include a deal of flexibility in how they might be installed. Surplus
space was added into the design of the internal chamber of the model
\textbackslash FIGURE, but there was still a limit given the structural
and aesthatic requirements.

\subsection{Keyboard}\label{keyboard}

\subsubsection{Anatomy of Harpsichord
mechanics}\label{anatomy-of-harpsichord-mechanics}

\begin{itemize}
\item
  simple
\item
  aesthetics
\item
  split workload
\item
  parts of harpsichord

  \begin{itemize}
  \item
    jack

    \begin{itemize}
    \item
      quill
    \item
      jack body
    \item
      pivot
    \item
      tongue
    \end{itemize}
  \item
    key
  \item
    body
  \end{itemize}
\end{itemize}

\subsubsection{Keyboard Construction}\label{keyboard-construction}

The desicion was made to design the elctronic sna dthe model keyboard in
tandem. Allow for quicker development time and iteration on the idea.
Likely the first implementation would need re-designed and redesiginng
the electronics to fit the form of the keyboard was more cost effective
than building or carrying out major modifications to the model.

The model was designed to have two internal chambers a front and rear.
The front and back that was xxx by yyy by 370mm between the faceplate
and the jacks. The TONGUE THAT GOES INTO SLOT divides the chamber though
a small gap between the under side of the soundboard and the key guides.
The key guide needs to be deep enough to allow for the full movement of
the key and stop the key becoming unseated. The chamber then extends to
a support wall which divides the front and rear.

Discussion of spatial constraints and affects on electyronics design and
installation is covered in \textbackslash section\{setup\}

The rear chamber was desigbned to house comonents for audio processing,
which is discussed further in \textbackslash section

\footnote{https://github.com/Nemus-Project/haptic-harpsichord-nime-2025/issues/Nemus-Project/haptic-harpsichord-nime-2025/images/disegno.pdf}

For the initial prototyping stage, modifying an existing harpsichord was
considered though ultimately discarded. An existing instrument would
have provided a test for scaling the electronics, but internal
measurments and layout would have been too different. A harpsichord that
was economic enough to be deconstructed would also have been of a more
modern mechanical design and not of the design that was being recreated
with the model. Prototyping stage would likely be invasive and damage or
heavy modification likely. As such it was not a good use of funds as a
modern instrument fit for destruction did not offer benefits that the
small model with easy internal access did not already provide.

\subsection{Building Process}\label{building-process}

PARAGRAPH FROM ROBERTO

\subsection{Prototyping}\label{prototyping}

\subsubsection{3-Key Prototype}\label{key-prototype}

\begin{itemize}
\item
  proof of concept
\item
  available on loan from museum
\item
  easy access to internal structure
\end{itemize}

\subsubsection{49-Key Prototype}\label{key-prototype-1}

For the final interface a 49-key (4-octave) was constructed by luthier
Roberto Livi. The keyboard is in the same style† as the 3-key model with
to jacks per key.

The keyboard is fully enclosed and internally has two chambers, front
and rear, in which the electronics could be installed
\textbackslash figure(underside 49-key).

Keyboard to be used for exhibition at san colombano where it would be
available to play by general public.

Timescale: from initial discussion to delivery of interface to the NEMUS
lab was 8-months. Active assembly of the instrument was 3 months

Constructed From:

\begin{itemize}
\item
  for exhibition
\item
  problems of scale
\item
  problems of timescale
\end{itemize}

\subsubsection{Sensor Criteria}\label{sensor-criteria}

A set of criteria was drawn to which the final sensor system would have
to satisfy. This criteria was used as a means of evaluating the
feasability throughout the prototyping phase.

\begin{itemize}
\item
  Non-invasive: The sensor system should not require major modifications
  to the interface on which it is installed.
\item
  Low Latency: Reading sensors and then triggering an output should have
  a low latency. Using {[}paper on latency{]} a target was set as X ms
  with an upper limit set to 25ms beyond which the sensors would be
  deemed unusable
\item
  Reliable: Data obtained should be reliable and accurate enough not to
  result in false positives or false negatives without extensive
  filtering
\item
  Scalable: The system should be scalable both economically and
  temporally. Given the open soucre nature of the porject, it could not
  be prohibitvely expensive to carry out. Also any system that would
  work on a single key would have to scale financially to potentially 50
  or more. The time required to install and calibrate the sensors would
  also have to scale.
\item
  Expandable: Any sensor system should be functionally expandable to
  allow for the usage of the data for control over other parameters.
\end{itemize}

\subsubsection{Triggering a note}\label{triggering-a-note}

For the first iteration the data from the sensor would simply be used to
test when a threshold had been passed. On passing the threshold a
message would be sent to trigger the playback of a corresponding note.

The message format covered in \textbackslash section

\section{Related Work and Motivation}\label{related-work-and-motivation}

\begin{itemize}
\item
  \cite{Timmermans2020}
\item
  \cite{McPherson2013} \cite{McPherson2019}
\item
  \cite{Mudd2013}
\item
  \cite{Fritz2017}
\end{itemize}

\section{Hardware Design}\label{hardware-design}

The hardware was designed with the xx in mind.

Using a similalr apporach to @HapticKey the first step was to fabricate
or procure a model of a single working key. San Colombano provided a
model 3-key of a harposchord mechanism by Graziano Bandini
\textbackslash FIGURE. The 3-key model was used a test-bed for potential
sensors while designs were drawn for a full-scale model. The
cross-sectional nature of the model meant it was easy to fit and remove
sensors. Small size meant that it would also be easy to transfer between
workshops.

3-key model also had a sliding jack rail which allowed for easy
disengagement of the physical mechanical response, which allowed for
some preliminary tests on the presence and absence of mechanism.

\subsection{Failed sensors}\label{failed-sensors}

A number of sensors were tried before the final system was decided upon.
An inertial measurment unit (IMU) was mounted to a single key which
contained a 6 degrees of freedome acceleromoetr and gyroscope. There was
a consistent set of conditions to trigger sound, but this was not easily
differntiable when other keys were played. The fitting time and cost of
a single IMU could b reduced when purchased in bulk but was far greater
than other systems considered.

A reed sensor and a hall sensor were mounted and magnets embedded within
the jacks. Adjusting the hieight of the sensor provided a reliable means
of detecting a threshold, inparticular the hall sensors which could be
partially calibrarted physically and also in softwrae. The time taken to
calibarte a single key was time consuming. Ideally each sensor would be
placed on a thread for adjustment, but given the constraints on the
internal space it was decided any adjustment of heights would likely
mean an unnacceptable length of time to install and calibrate. Embedding
magnets into the jacks was also did not satisfy the criteria of being
scalable and non-invasive.

Light dependant resistor mounted to the jack rail. as jacks were pressed
less light would be let through. Usable data but very dependant on
lighting conditions. Modifying the jack also did not satisfy the
non-intrusive criteria and the placement outside the internal chamber of
the model meant hiding electronics would be more difficult.

Finally, force sensitive resistors placed under the jack of each key was
considered but was never implemented. The sensors were to be placed
under each jack either above or below the padding on which the bottom of
the jack rests. Further discussion of this apporach is in
\textbackslash section.

\section{Electronics}\label{electronics}

Design of electronics carried out in tandem with construction of full
model. Since the Electronics were designed at a distanced to the final
full model some flexbility was required in the design. Design decisions
that were not as optimal but would allow for change in wiring and
fitting should there be any issues during installation.

Electronics designed so that they would not require any equipment
outside what can commonly be found in a maker space at a univeristy.

The equipment required including soldering suitable for throughole
technology (THT) components and 3D printing.

Laser cutting and CNC routing fascilities were used during the research
for the project but not required for the final implementation.

The idea behind this was to allow for the project to be reasonably
recreated without procurement of any additional specialist equipment.

NEMUS lab under construction during the first section of the project.
Facilities available at the maker spaces of Bologna (Alma Labor) and
Edinburgh (uCreate) used for what was deemed reasonable assumption in
terms of available equipment.

\begin{itemize}
\item
  Over arching themes

  \begin{itemize}
  \item
    simplicity of design

    \begin{itemize}
    \item
      Limited resources
    \item
      easily reproducible
    \end{itemize}
  \item
    availability and lead times
  \item
    iteration and design from distance
  \end{itemize}
\end{itemize}

\subsubsection{QRE1113}\label{qre1113}

Following from the work in \cite{McPherson2013, McPherson2019} we tested
a system using the Fairchild QRE1113 \textbackslash datasheetfigue. The
QRE1113 is a combination infrared LED and phototransistor sensitive to
IR light. The phototransistor of QRE1113 provides data on how much light
is present and in particular how much light is being relfected from a
surface. Since refelected light will be proportional to the distance of
a surfcae it is a good, close proximity, distance sensor. Distance is
the way in which the instruments in \cite{McPherson2013, McPherson2019}
function as well the Moog Piano Bar from which it too inspiration.

The limitation for taking a distance approach is that there is not one
but 2 jacks per key. For each jack to be measured indepeedntly the data
would need tp com fropm the jacks themselves.

They \emph{could} be used and individual trigger points set in
calibration, but this creates new problems.

\begin{itemize}
\item
  The triggering would need to be bespoke for every key as fluxuations
  in the dimensions of every piece mean that the threshold would be
  different.

  \begin{itemize}
  \item
    Moving a problem into additional time in the calibtraryoin setup
    phase.
  \end{itemize}
\item
  Remebering the constraint that system isshould noyt fix problem, there
  would be no easy way to tell if a jack had nbot reseated.

  \begin{itemize}
  \item
    In playing a harpihchord when the jack does not go to its rest
    position, not note would play. This fault should be carried over
    intp the final instrument.
  \end{itemize}
\end{itemize}

Another common application of the QRE1113 is within line following
robots, where an array of QRE1113s ar used in conjunction with black
tape along which some motorised system can following

Applied to the jacks was a greyscale gradient printed the length full
travel range of the jack.

For the first iteration this was performed on an inkjet printer and
double-sided tape. For the final iteration the stickers were printed on
(Coala, 1D 100 Gloss P, gloss, white, permanent adhesive, 300g/m2, 100
µm, 1370mm x 50.00m, Solvent/Latex/UV) using a HP Latex 115 and then
cutout with a Summa 150 roll cutter. The printer and roll cutter greatly
reduced the manual labour for cutting stickers meaning the process would
scale from the 3-key to the 49-key model.

An initial sticker design was used that contained only the gradient. It
was found that the jacks wer sticking in their slots andthe edges on the
stickers easily peeled around the printing The sticker was redesigned to
be the full length of the jack which avoided the problem of peeling. The
larger are also mean the stickers adhered betterto the jack Some
trimming was still required of each sticker to avoid them adding extraa
resistence to the jack's movement.

During testing it was found that their was a considerable amount
cross-talk bwteen adjacent sensors. This meant the reading from key
would be affected by the movement of the key next to it. The cross-talk
was assumed to be as a result of light reflecting from other jacks.
Given that IR is used this was difficult to confirm by sight. A set of
baffles was designed for the pcb that would wrap round the jack and stop
light from other jacks or sensors. Baffles were printed on BAMBU X1 3D
printers using PLA plastic. A dark pigmented PLA was used to avoid IR
light simply being diffused. After the baffles were fitted the
cross-talk was entirely eliminated.

\begin{itemize}
\item
  previous use
\item
  problem of what to sense
\item
  satisfied both criteria and design restrictions
\item
  usage

  \begin{itemize}
  \item
    Sweet spot is around 6mm distance
  \end{itemize}
\end{itemize}

\subsubsection{sensor surface}\label{sensor-surface}

\begin{itemize}
\item
  linear gradient
\item
  on Jack
\item
  printed and cut on vinyl {[}find name{]}
\item
  cut to travel size
\item
  then to jack size
\item
  jacks varnished on side
\item
  alternatives

  \begin{itemize}
  \item
    CNC plywood and mdf

    \begin{itemize}
    \item
      problems
    \end{itemize}
  \item
    laser cut plywood and mdf

    \begin{itemize}
    \item
      problems
    \end{itemize}
  \item
    3D printed jack body

    \begin{itemize}
    \item
      problems
    \end{itemize}
  \end{itemize}
\end{itemize}

\subsection{Arduino}\label{arduino}

\subsubsection{Nano BLE}\label{nano-ble}

Arduino Nano BLE was chosen for a few reasons. The Nano form factor was
appealing as it would allow for easy changes between chipsets without
requiring any rewiring and thus no new PCBs would need to be created if
a better alternative was found later in the project. Another benefit to
the Arduino Nano boards was that typically had native-USB meaning the
could be programmed as a hardware USB MIDI device directly.

The Nano BLE specifically was used in the initial stages for the
integrated IMU.

When it was finalised that the prokect would proceed using the QRE1113 a
number of microcontroller units (MCU) were condsidered.

The MCUs considered were the Arduino Nano 33 IoT, Arduino Nano 33 BLE,
Arduino Nano ESP32 and STM32 NUCLEO-L031K6.

A rough performance test was later carried out using an STM32
NUCLEO-L031K6, a Nano ESP32 and the BLE. The Nano BLE was found to be
more performative. To execute a single sensor read cycle of the firmware
the STM32 took 40ms, and both the Nano ESP32 and Nano BLE required 11ms
on average. The Nano ESP32 had high variability so it was decided to
continue with the Nano BLE. Further discussion of using ESP32 continues
in \textbackslash section\{Going forward\}

BLE functionality also meant if the cable connection between the nano
and computer running audio synthesis was impractical, BLE MIDI was
available as a fallback. The Nano BLE also has a 12-bit DAC, which
provided a contingency should the default 10-bit not be sensitive enough
for the signal from the sensors.

Initially a Nano IoT was used for both WiFi and Bluetooth Low Energy
functionality, but it was found that 2 ADC channels were in fact not
useable and would have required additional multiplexers.

\begin{itemize}
\item
  Nano form factor small
\item
  allows for easy changes in chipset
\item
  ``Native USB''
  \footnote{https://learn.adafruit.com/dude-where-s-my-com-port/native-usb-boards}
\item
  BLE Model

  \begin{itemize}
  \item
    Performative
  \item
    BLE capabilities for MIDI BLE transmission
  \item
    All analog pins available
  \item
    available at the time
  \item
    12-bit DAC
  \end{itemize}
\end{itemize}

\subsection{EEPROM / NVRAM / FRAM}\label{eeprom-nvram-fram}

There would need to be the ability to store values on non-volatile
memory so that the electroincs could be powered-down without losing
data.

Adjusting the values through firmware would not have been flexible or
practical. Instead an Ferroelectric RAM unit was chosen and connected
using an SPI interface. NVRAM or EEPROM would have sufficed and the FRAM
was primarily chosen for it's ready availability.

Most microcontrollers have some variety of non-volatile program memory
that can be read and writ accessed but in most cases this memory is
cleared during compilation and re-flashing. The small cost was far
outweighed by the potential for data loss if using onboard RAM.

\emph{It is worth noting that the FRAM model chosen stopped being
available during the project, meaning any future iteration would need to
adjust for another board in firmware}

The firmware was structure in a way that the interface could be changed
so long as the code required for any other board would comply with
function arguments and outputs. Worth noting the fragility and
volatility when designing this kind of project and how easily it is for
the project to become broken.

SD card would provide an easier way to interact with calibration data.
The posiiton of the controller board did not permit easy access.
reliability and speed was preffered over storage capacity.

\begin{itemize}
\item
  On board available, but volatile
\item
  retained across firmware changes
\item
  small,
\item
  could have been sd card or anything else
\item
  available
\item
  interfaced via rotary encoder
\end{itemize}

\subsection{RGB LEDs}\label{rgb-leds}

for problems of scale:

Addressing the problem of scale of identifying which key's threshold is
currently being edited.

The LEDs also allowed for easy identification for keys that were outside
of expected values and when a key was triggered. This allowed for visual
identification of malfunctioning or uncalibrated keys.

The LEDs are on the same PCB as the sensors and are hidden when the
nameboard is in place.2024

Would sugggest that this is a necessity of this kind to provide a means
to identify and diagnose LEDs with state.

TLC5940 or other PWM driver such as in \cite{McPherson2013} added
complexity to assembling, but were avoid in favour of Addressable LEDs
with intergrated driver. The intergrated LEDs share the same data line
which has the limitation where a fualty LED will stop all following LEDs
from working. An individual PWM driver per PCB would allow avoid
situations where failutre cascades, but would add complexity in assembly
as the parts are typically SMD. Also, given the LEDs are fixed to the
board and could not be repaired in situ, the solution would be to
replace the PCB regardless of which approach was taken. Given the LEDs
were only used in debugging and calibration and otehrwise would not be
visible, a simpler design approach was taken.

\begin{itemize}
\item
  locating keys
\item
  displaying key state
\end{itemize}

\subsection{Multiplexing}\label{multiplexing}

CD4051BE Texas Instruments

problem of scale

For the 3-key model, individual jacks could be connected to one of the 8
available ADC pins.

The BLE Nano has 8 ADC pins connected to one ADC channel via a
multiplexer

Multiplexer added to expand to the

Transistor / Op-amp circuit could have been used similar to the one in
\cite{McPherson2013}.

Benefits were keeping with THT based designed for easy assembly.

An 8-channel multiplexer was chosen as the model has 49-keys and a
division of 7 PCBs each containing 7 sensors

The multiplexer reduced 7 signals to a single pin. Three pins are
required pins for addressing the muliplexewrs and all PCBs are addressed
simulatenously.

\begin{itemize}
\item
  limited ADC channels
\item
  ADC channels already multiplexed
\item
  simple
\item
  8 channel
\item
  THT
\item
  7 into 49

  \begin{itemize}
  \item
    7 PCBS
  \item
    each with 7 sensors
  \end{itemize}
\end{itemize}

\subsection{PCB design}\label{pcb-design}

\begin{itemize}
\item
  EAGLE PCBs designed using EAGLE. Worth noting that AutoCAD has
  deprecated EAGLE and the programme will no longer be supported from
  2026 Project files will need to be transitioned to another standard,
  likely the KiCAD programme.
\end{itemize}

trade off between limiting number of boards required and allowing
compensation for variation in pitch.

Smaller PCBs means that spacing could be adjusted to accomodate the
changes in pitch between jacks. It also allowed for a modular approach
where any PCB could swapped for minimal cost.

\subsubsection{Sensor board}\label{sensor-board}

The 49 QRE1113s are divided across 7 PCBs each containing

\begin{itemize}
\item
  x7 QRE1113
\item
  x7 100R
\item
  x7 10kR
\item
  x1 CD4051BE Multipexer
\item
  jack / sensor pitch by board
\item
  7 boards adjust for pitch/spacing error
\item
  modular, replaceable
\item
  orientation problems

  \begin{itemize}
  \item
    legs break easily
  \item
    end mount like A.P.Mc. time consuming
  \item
    problem

    \begin{itemize}
    \item
      milling out keys
    \end{itemize}
  \end{itemize}
\item
  designed from distance
\item
  striped lines across pcbs
\item
  problems

  \begin{itemize}
  \item
    additional terminals uneeded resulted in lots of additional work
  \item
    long cabling increase risk of breaking
  \end{itemize}
\item
  solutions

  \begin{itemize}
  \item
    striping all adc lines with FFC
  \item
    solder pads to select channel
  \end{itemize}
\end{itemize}

\subsubsection{controller board}\label{controller-board}

Controller board designed around Arduino nano form factor. The Nano is
connected through headers to allow easy switching between chips. Should
another chip in a Nano for factor be found it could supplant the current
Nano without any need for a redesign.

The first version of the board esclusively used 2.54mm pitch headers.
The reasons for this were flexibility when creating cable looms and to
leave open the possiility of rewuring the PCB.

This was the case for the LEDs as they initially wer to be powered on
5V. It was discovered that the 3v regulator of the nano was sufficient
to power all LEDs at a lower brightness while reducing current draw.
Since brightness was not importnat.

The 2.54mm pitch headers had two major drawbacks. First, the orientation
of the cables could be reversed, which in the best case scenario would
simply mean the system did not work correctly and in the worst case it
could potentially damaged some components.

Secondly, the headers do not connect securely and wait of the cabling
was enough for components to become disconnected as the harpsichord was
handled.

To address this, once the wiring for the controller board was finalised
a new design was drafted this time using JST-PH connectors. No
commercial cables were immediately available in the configuration and
length required, as such a cable loom needed to be made which meant
crimping cables by hand. A crimping tool from JST is prohibitively
expensive compared to the cost of other components in the project
\footnote{from RS https://uk.rs-online.com/web/p/crimp-tools/6880877
£508 (inc. VAT)}. The crimping tool also appeared to be uncommon and was
not readily available in the workshops at the projects disposal.
Eventually a tool was found, but an alternative approach was also
researched in order to maintain the criteria of accessibility.

A locator was adapted from a creative commons stl model
\footnote{https://www.thingiverse.com/thing:1646016} that allowed a
generic crimping tool to be adapted for use with JST-PH crimps†
\textbackslash figure\{\}. The locator was printed using a Bambu X1 3D
printer. A second alterntaive was assembling cables from other sets that
were at hand.

\begin{itemize}
\item
  mount for nano to allow switching
\item
  initially push fit 2.54mm header

  \begin{itemize}
  \item
    not a secure fit
  \end{itemize}
\item
  JST-PH used

  \begin{itemize}
  \item
    crimping required custom cables

    \begin{itemize}
    \item
      could be assembled from other cables
    \item
      length limited
    \end{itemize}
  \item
    prohibitevly expensive to crimpers
  \item
    3D print helper to allow for crimping with generic crimpers
  \end{itemize}
\end{itemize}

\subsection{3D printing}\label{d-printing}

Access to a 3D printer is was vital to the success of the project. Given
the proliferation of 3D printing, particular in maker spaces in
universities it was deemed an unreasonable requirement. 3D printed parts
include the support brackets for PCBs, the baffles for the sensors,
locator for crimping JST-PH crimps and a mount for the MacMini used for
audio synthesis.

The printer used throughout the project was a Bambu X1-Carbon using
recycled PLA filament.

Support brackets and baffles were made for the project while the MacMini
bracket and JST locator were sourced from creative commons models.

Any colour of filament can be used, though it is advised to use a dark
coloured filament for the baffles.

See each corresponding section for more detail on 3D printed components.

\begin{itemize}
\item
  https://www.thingiverse.com/thing:1646016
\item
  https://www.thingiverse.com/thing:2886460
\item
  Supports

  \begin{itemize}
  \item
    Push fit
  \item
    adjustable for best average placement
  \end{itemize}
\item
  Blinkers

  \begin{itemize}
  \item
    reduce noise from other senors
  \end{itemize}
\item
  mounts

  \begin{itemize}
  \item
    MacMini
  \end{itemize}
\end{itemize}

\subsection{Power}\label{power}

All electronics draws around 1.1amps of current at 5V with some
fluctuation during boot.

Arduino BLE has some functionality disabled to limit current draw.

Ideally the unit could be bus powered from USB.

The sensors are in a state of constantly drawing power. It is possible
that powering only when data is required could be reduce power enough.

Current designs have a separate MIDI device for each jack row. Combining
the devices and also reducing current draw to below the common 0.5 amps
for a USB port would certainly be a challenge.

\section{Firmware Design}\label{firmware-design}

Arduino platform chosen for wide adoption, relative ease of use and
library support for components.

Firmare checks components are connected then reads the FRAM to see if
threshold data is already present. The rotary encoder and it's momentary
switch are polled for any change in state. A key is selected with the
rotary encoder while the momentary switch is used for changing between
states for key selection and threshold adjustment A double click of the
rotary encoder writes the current threshold values to FRAM. The sensors
for the current multiplexer channels are polled. If any sensor reads
above the threshold when it was previously below a MIDI On message is
sent. For the reverse case where the value was below the threshold when
it was previously above, a MIDI Note Off message ois sent.

\begin{itemize}
\item
  Arduino platform

  \begin{itemize}
  \item
    library support and portability
  \item
    dependancy management
  \item
    wide spread use
  \end{itemize}
\item
  core algorithm

  \begin{itemize}
  \item
    check components connected
  \item
    READ data from FRAM
  \item
    select key with rotary

    \begin{itemize}
    \item
      click to edit, double to save
    \end{itemize}
  \item
    current threshold and reading sent over serial and plotted
  \item
    when over threshold send MIDI note
  \end{itemize}
\end{itemize}

\subsection{Project structure}\label{project-structure}

The project is separated into three repositories encapsulating

\begin{itemize}
\item
  Firmware
\item
  Keyboard CAD Data
\item
  PCB CAD Data
\end{itemize}

\subsubsection{Firmware}\label{firmware}

Firmware repository contains all firmware for the full and smaller
models and components.

\subsubsection{Keyboard}\label{keyboard-1}

The Keyboard repository contains all plans for fabricating the keyboard,
measurements and 3D models of the jacks.

\subsubsection{PCB}\label{pcb}

PCB repository contains the EAGLE files for creating the sensor and
controller boards.

Given the regular spacing of components, scripts for automatic placement
of components based on parameter of jack pitch spacing and number of
sensors per jack.

\section{Implementation}\label{implementation}

\subsection{Installation}\label{installation}

The electronic components were partly pre-assembled at the university of
Edinburgh. Though all measurments were known ahead, enough flexibiliyty
was needed to allow for proboems to solved in situ.

The final assembly took place at the NEMUS lab at the univerity of
bologna over the final week in October 2024 for delivery at San
Colombano for exhibition in the first week of November in 2024.

A misreading in the original clearance between the keys and the PCB
meant that the keys wer block entirely from moving. A 10mm by 15mm
section was milled out of each lke \textbackslash FIGURE to create extra
clearance. The milling required the removal, shortening and refitting of
the leather pads for the jacks \textbackslash FIGURE. The extra work
added an extra day to the full install.

A redesign of the PCBs meant that removing the material would no longer
be necessary.

The PCBs shared power and data lines for components with the exception
of the output from the multiplexer. Ribbon cable was cut to size to
allow for some room to readjust the distance of the sensors from the
jacks. The data signals from each PCB were connected using a separate
cable loom. Soldering cables and inspecting each join required another
day work. This did allow for readjustments to the positon of the
controller board Measurements of the rear chamber did not match with the
MacMini that was to be used a different layout was decided for fitting
the controller board/ Cutting cables to size allowed for a great deal of
felxibility but at the cost of time. After the first install it was
clearer what constraints were in place for fittig and an alternative
wiring system using flat flex cable was implement (discussed in
\textbackslash section)

Fitting PCBs required the removal of all keys. The support brackets were
assembled and screwed into the the underside. With the harpsichord on
it's back the sensors were tested to ensure an expected range.
Self-tapping round head screws were used to to attach the brackets. The
screws were not over tightened to allow for brackets to move. The
brackets were not easily adjusted once the key were replaced. Adjusted
to a best average with subsequent adjustment carried out on the
microcontroller.

As the PCBs were fitted the jacks had the gradient stickers applied to
them. The jacks had been coated with a varnish on the short side to
provide a better surface for the stikcers to adhere to. After the
stickers were applied excess material was trimmed from each jack. The
jack were then check that they moved freely in their slot.

After fitting the PCBs preliminary calibration was carried out. Before
the keys were reffitted the jacks wer put back in place. The data from
each key was confirmed to be in a similar range. Any key showing
anomalous dat had it's jack removed and re-inspected. The PCB fitting
process and refitting required another day as faults were found in
wiring or misalignement of the stickers. After the sensors were
confirmed to be functioning the keys were put back.

The calibration workflow was with Dr.~Craig Webb who was unfamiliar with
the system. Adjustments were made to the firmware to to make it more
intuitive for the user. Addressable LEDs became vital as it allowed for
easy identification of the key being calibrated. They were also used to
identify any keys whose current reading was beyond what was expected.

The rotary encoder was used as the interface to select a key, adjust
it's threshold and update the Ferroelectric RAM (FRAM) with new values.

A process was implemented where the key was selected by playing, but
this caused problems in situations where the data was unusual due to
misalignement of the sensors

This would likely be a better system for when the keyboard was fully
operational

The thresholds were calibrated visually utilising the serial plotter of
the Arduino IDE and MIDI was confirmed using the open source MIDI
Monitor \footnote{https://github.com/krevis/MIDIApps}.

\begin{itemize}
\item
  partial pre-assembly
\item
  handsolder components

  \begin{itemize}
  \item
    analog lines
  \end{itemize}
\item
  connecting supports
\end{itemize}

\subsection{Context}\label{context}

Designed as part of an exhibition for the museum San Colombano in
Bologna. Instruments is to be played by general public as a means of
interacting with items in the collection that are no longer in playing
condition. Audience of one who listens via headphones.

The strings of the interface are damped with felt strips. The damping
did change the tension and thereforew the feel of the strings so
adjustments needed to be made to the tuning to accomodate for this.

\begin{itemize}
\item
  in museum context
\item
  strings damped
\item
  headphones
\item
  interfacing with kontakt spitfire library
\item
  display

  \begin{itemize}
  \item
    secure storage and access
  \item
    interface to the public
  \end{itemize}
\end{itemize}

\subsection{Calibration}\label{calibration}

Finer calibration carried out with the guidance of Catalina Vincens as
an expert harpsichord performer

\begin{itemize}
\item
  sensor surface

  \begin{itemize}
  \item
    printed gradient
  \end{itemize}
\end{itemize}

\section{Conclusion}\label{conclusion}

\begin{itemize}
\item
  scale
\item
  time scale
\item
  working at distance
\item
  limitations

  \begin{itemize}
  \item
    latency
  \item
    deprecation

    \begin{itemize}
    \item
      software

      \begin{itemize}
      \item
        EAGLE
      \end{itemize}
    \item
      hardware

      \begin{itemize}
      \item
        FRAM
      \end{itemize}
    \end{itemize}
  \end{itemize}
\end{itemize}

\section{Going forward}\label{going-forward}

\begin{itemize}
\item
  Velocity, aftertouch
\item
  second jack row
\item
  reducing latency
\item
  alternative sensor surfaces
\end{itemize}

A limitation in the design is the variablility in positioning of tage,
especially when they need to be reapplied. A jig was designed for
applying tags, but could not be printed in time for installation. Even
with a jig, some tags had to be radjusted by hand regardless. Long term
this hasd impliccation on how to maintain the interface.

Taken a different approach to what surface is used. Preliminary test
were taken using Laser Cutting and CNC to fabricate Jack bodies. Laser
cutting results in a loss of material that has too much ariablity. The
edge o f the jack body will be charrred. The material also needs to be
reliable. on the advice of two luthiers, roberto livi and Jonathan Santa
Maria Bouquet at Edinburgh, Plywood and MDF have problems with
variations in humidity, tear-out and delamination.

Some tests were carried out using a notched jack body that was 3D
printed. The notch means that as the jack raises, the forward material
acts like a shutter. No reasons that this could not be CNC though a
variety of materials would be needed. Traditional material used is xx
though need to test of the material is naturally reflective enough or if
a coating would beed to be applied.

Outputs of the NEMUS project will also include numerically simulated
non-linear plucked string models with which the device can interact. A
bespoke sound synthesis engine to leverage data from both jack rows
which will also be open-sourced. Design for instrument to be compatible
with standard MIDI instruments as well software designed for the full
data output.

A second interface designed more closely to the Trasuntino style
instrument commisioned by the NEMUS project with Roberto Livi. Using
lessons learned for the first iteration the second can make better
acommodations for internal electronics. Interface to be used for a
performer study and used for performance with a numerically simulated
sound syntheiss engine.

Testing current sensors in conjunction with force sensitive resistors to
research if a impulse response for the pluck can be extrapolated.

Future novel designs for an interface that would utilise velocity and
aftertouch messages in software. This would be applicable for numericla
simulations of strings. A bespoke piec e of sowftare would also allow
for direct mapping of jack rows to registers. There is enough data to
implemenet functionality for velocity and aftertouch MIDI messages and
simply needs to be activated in the firmware. (Hyperinstruments
\cite{nime2024_20})

seat vibrations \cite{MusicalHaptics2018_07}

\begin{itemize}
\item
  ``there is a general connection between vibrations and the perceived
  quality of music reproduction''
\item
  ``influence of vibrations on loudness perception at low frequencies''
  Evaluation
\end{itemize}

\subsection{improvements}\label{improvements}

\begin{itemize}
\item
  ESP-IDF

  \begin{itemize}
  \item
    latency half
  \item
    workload high
  \end{itemize}
\end{itemize}

A current limitation is the latency in triggering MIDI output. Currently
the latency is around 20 ms, but should ideally be closer to 5ms.

An obvious approach would be merely changing the microcontroller. A test
was carried out using an Arduino Nano ESP32 which is a NORA-W106, a
module with a ESP32-S3. The benefit of this microcontroller is that it
would allow simply the board to be swapped and the rest of the PCB
designs would be unaffected.

Using the Espressif development framework ESP-IDF and follwong the same
logic as the original firmware source, the latency was reduced to around
10ms. ESP-IDF allows access to functionality of the ESP32 that is not
exposed in the Arduino libraries. It is possible that the 10ms latency
could be improved upon but there would be a cost in the form of
develoment time. Consideration would need to be made on how to port
existing code or whether to completely re-write the firmware.

\begin{itemize}
\item
  avoid soldering cabling
\item
  power indicators on pcbs

  \begin{itemize}
  \item
    debug if problem is power related Add power small power LEDs to the
    each PCB. It was not common for a board to not be functioning and
    thye result to be that the power weas not correctly connected.
    Especially when the ower is connected via somethgin such as an FFC
    cable where it is easy to plug-in the wrong way esepcially when
    fitting is to be carried out by someone unfmailiar wiyth the system.
    Power LEDs would quickly confirm whether the problem was power
    related. They would have to be small enough and low enough current
    draw to avoid glare or being mistaken with other indcator LEDs To
    continue using THT components, the choice would have to be moade
    carefully.
  \end{itemize}
\end{itemize}

\section{Ethical Standards}\label{ethical-standards}

To ensure objectivity and transparency in research and to ensure that
accepted principles of ethical and professional conduct have been
followed, authors must include a section ``Ethical Standards'' before
the References. This section should include (if relevant): information
regarding sources of funding, potential conflicts of interest (financial
or non-financial), informed consent if the research involved human
participants, statement on welfare of animals if the research involved
animals or any other information or context that helps ethically situate
your research. For help with the ethics section, feel free to ask on the
NIME forum: \url{https://forum.nime.org}.
