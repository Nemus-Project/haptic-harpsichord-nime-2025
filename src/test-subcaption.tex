\documentclass{article}
\usepackage{graphicx,subcaption}
\begin{document}

    % \begin{figure}
    %     \centering     %%% not \center
    %         \subfigure[Figure A]{\label{fig:a}\includegraphics[width=60mm]{example-image-a}}
    %         \subfigure[Figure B]{\label{fig:b}\includegraphics[width=60mm]{example-image-b}}
    %     \caption{my caption}
    % \end{figure}

    \begin{figure} \centering
        \begin{subfigure}[b]{\linewidth}
            \includegraphics[width=60mm]{example-image-a}            
            \label{fig:a}
        \end{subfigure} %
    
        \begin{subfigure}[b]{\linewidth}    
            \includegraphics[width=60mm]{example-image-b}
            \label{fig:b}    
        \end{subfigure} 
        \caption{my caption}
    \end{figure}


    \begin{figure}
        \begin{subfigure}{0.65\textwidth}
          \includegraphics[width=\textwidth]{example-image-a}
        \end{subfigure}
        \hfill  % NOTE1: hfill moves horizontally stacked objects as far apart as it can
        \begin{subfigure}{0.32\textwidth}
          \begin{subfigure}[t]{\textwidth}
            \includegraphics[width=\textwidth]{example-image-a}   
          \end{subfigure}\\

          
          \begin{subfigure}[b]{\textwidth}
            \includegraphics[width=\textwidth]{example-image-b}
          \end{subfigure}\\

          
          \begin{subfigure}[b]{\textwidth}
            \includegraphics[width=\textwidth]{example-image-a}
          \end{subfigure}
        \end{subfigure}
    
        \caption{figure caption}
  \end{figure}
    
\end{document}